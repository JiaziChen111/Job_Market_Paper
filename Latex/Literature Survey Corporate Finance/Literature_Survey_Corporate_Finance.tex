\documentclass{article}
\usepackage[utf8]{inputenc}
\usepackage[english]{babel}
\usepackage[a4paper,top=3cm,bottom=3cm,left=3cm,right=3cm,%
bindingoffset=0mm]{geometry}
\usepackage{amssymb}
\usepackage{amsmath}
\newtheorem{prop}{Proposition}
\newtheorem{lemma}{Lemma}
\newenvironment{proof}[1][Proof]{\begin{trivlist}
\item[\hskip \labelsep {\bfseries #1}]}{\end{trivlist}}
\newcommand{\qed}{\nobreak \ifvmode \relax \else
      \ifdim\lastskip<1.5em \hskip-\lastskip
      \hskip1.5em plus0em minus0.5em \fi \nobreak
      \vrule height0.75em width0.75em depth0em\fi}
\usepackage{tikz}
\usepackage{graphicx}
\usepackage{rotating}
\usepackage{float}
\linespread{1.3}
\raggedbottom




%
\font\reali=msbm10 at 12pt
% subsets of real numbers
\newcommand{\real}{\hbox{\reali R}}
\newcommand{\realp}{\hbox{\reali R}_{\scriptscriptstyle +}}
\newcommand{\realpp}{\hbox{\reali R}_{\scriptscriptstyle ++}}
\newcommand{\R}{\mathbb{R}}
\DeclareMathOperator{\E}{\mathbb{E}}
%

\title{Related Literature}
\author{Marco Brianti}
\date{A.Y. 2018/2019}

\begin{document}
	\large{

\maketitle

\tableofcontents

\section{Almeida, Campello, and Weisbach (2004) - JF}

Two important areas of research in corporate finance are the effects of financial constraints on firm behavior and the manner in which firms perform financial management. These two issues, although often studied separately, are fundamentally linked.  As originally proposed by Keynes (1936), a major advantage of a liquid balance sheet is that it allows firms to undertake valuable projects when they arise. However, Keynes also argued that the importance of balance sheet liquidity is influenced by the extent to which firms have access to external capital market. If a firm has unrestricted access to external capital (financially unconstrained) there is no need to safeguard against future investment needs and corporate liquidity becomes irrelevant. In contrast, when the firm faces financing frictions, liquidity management may become a key issues for corporate policy. Firms anticipating financing constraints in the future respond to those potential constraints by hoarding cash today. Holding cash is costly, nonetheless, since higher cash savings require reductions in current, valuable investments. Constrained firms thus choose their optimal cash policy to balance the profitability of current and future investments.

Their basic model is a simple representation of a dynamic problem in which the firm has both present and future investment opportunities, and in which cash flows from assets in place might not be sufficient to fund all positive projects. Depending on the firm's capacity for external finance, hoarding cash may facilitate future investment. The model has three dates and firm has investment opportunity in both second and last period. Firm uses cash and borrowing to finance both investments. In this setup, firm is concerned only about whether or not to store cash in the first or second period to finance both investment opportunities. Financial constraints arise from the fact that only a part of future flows can be pledged as collateral and banks do not lend more than this partial amount. The crucial feature is that some firms have some limitations in their capacity to raise external finance, and that such limitations may cause those firms to invest below first-best levels. For a financially constrained firms, holding cash entails both costs and benefits. A constrained firm cannot undertake all of its positive projects, so holding cash is costly because it requires sacrificing some valuable investment projects today. The benefit of cash is the increase in the firm's ability to finance future projects that might become available. Optimal cash policies arise as a trade-off between these costs and benefits, both of which are generated by the same underlying capital market imperfection. \textbf{Proposition.} The cash flow sensitivity of cash (how much increase cash holding for an additional dollar of cash flow available) has the following properties: (i) positive for financially constrained firms; (ii) indeterminate for financially unconstrained firms.

They then test the model's main prediction using firm level data from COMPUSTAT. Their simplest test regresses current cash holding-to-assets changes on current cash flow-to-assets and Tobin's Q and size control. They then separate financially constrained firms over non-financially constrained using 5 different approaches as robustness check. According to descriptive analysis, constrained firms hold far more cash on their balance sheet. Moreover, the set of constrained firms displays significantly positive sensitivities of cash to cash flow, while unconstrained firms show insignificant cash-cash flow sensitivities. Thus, there are systematic differences between constrained and unconstrained firms in the way they conduct their cash policies, and that these differences are manifested along the lines suggested by their theory. Finally, they also show that during a downturn, constrained firms tend to stare relatively more cash from cash flow. This should happen because these periods are characterized both by an increase in the marginal attractiveness of future investments (when compared to current ones), as well as by a decline in current income flows. 


\section{Han and Qiu (2007) - Journal of Corporate Finance}

Why do firms hold a large percentage of cash and cash equivalents in their assets even though there is an opportunity cost associated with cash and cash equivalents? According to Keynes (1936) there are two major benefit of cash holdings. First, a firm can save a transaction costs by using cash to make payments without having to liquidating assets. Second, and possibly more important, a firm can reserve cash to hedge for the risk of future cash shortfalls; this is the precautionary motive for cash holdings. The precautionary motive for corporate cash holdings, however, has not been adequately modeled in the literature. Empirical results suggest that firms use internally generated funds to hedge against future cash flow uncertainty and to increase their cash holdings in response to increases in cash flow volatility. The objective of this paper is to provide an analysis of corporate precautionary cash holdings. They extend the theoretical model of Almeida, Campello, and Weisbach (2004) to analyze the corporate precautionary cash holding. In this model, a firm making decisions on investments over two periods cannot hedge future cash flow risk directly in the market but has to rely on cash reserves to hedge for future cash shortfalls. The objective of a firm is to maximize the present value of dividends payout to investors. Solving for optimal corporate cash holdings and investments in such a setting allows them to capture the key driving forces of the precautionary motive in corporate cash holdings, that is the limited diversifiability of future cash flow uncertainty and the inter-temporal trade-off between current and future investments.

They analyze the role of financial constraints on precautionary corporate cash holdings. A firm is financially unconstrained if it has enough financing capacity to make the first-best investments in both periods regardless of the realization of the future cash flow. The first-best investment in each period is determined at the point where the expected marginal return on investments is equal to the marginal cost of borrowing. Therefore, optimal future investment is independent of optimal current investment. Since a financially unconstrained firm has enough financing capacity to make the first-best investments, an optimal financing policy is not defined and perturbations of the system do not affect optimal level of investment and the indetermination of the financial policy. Therefore, there is no systematic relationship between cash holdings, investment levels and future cash flow volatility. 

However a financially constrained firm cannot make additional future investments without reducing current investment because it has exhausted all the external financing resources. Therefore, the firm can invest more in the future only by holding more cash and by reducing current investments. When the marginals return on investments are convex, an increase in future cash flow volatility makes the expected marginal return on future investments higher for given cash holdings. Therefore, an increase in future cash flow volatility leads the constrained firm to be more prudent, to increase cash holdings for more future investment by decreasing current investment. This precautionary motive of cash holdings creates a positive relationship between cash holdings and future cash flow volatility (uncertainty) and a negative relationship between current investments and future cash flow volatility (uncertainty) for financially constrained firms. Thus, if a firm is financially constrained, higher expected cash flow volatility induces it to increase its cash holdings and to voluntary reduce its current investment level because of the inter-temporal trade-off between current and future investments. To see details of the model see \textit{ACW model adaptation}.

They evaluate their theoretical prediction by investigating the cash holdings of a sample of publicly traded firms using Compustat quarterly from 1997 to 2002. They test the impact of cash flow volatility on their cash holdings for two different groups of firms: financially constrained firms and financially unconstrained firms. They find that the impact of cash flow volatility ona  fims's cash holdings dependes on a firm's financial constraint status. The financially constrained firm increases its cash holdings in response to an increase in cash flow volatility. In contrast, the cash holdings of financially unconstrained firms are not sensitive to cash flow volatility. Thus, the empirical evidence supports their theory. 


\section{Campello, Graham and Harvey (2010) - JFE}

In the fall of 2008, world financial markets were in the midst of a credit crisis of historic breadth and depth. In this paper, they provide a unique perspective on the impact of the crisis on the real decisions made by corporations around the world. One distinguish feature of their analysis is that they directly ask managers whether their firms are financially constrained. They start by documenting that the typical firm in the US sample had cash and marketable securities to about 15\% of total assets in 2007. Unconstrained firms are able to maintain this level of cash balances into late fall 2008. However, constrained firms burn through about one-fifth of their liquid assets over these months, ending the year with liquid assets equal to about 12\% of asset value. This evidence is consistent with the view that financially constrained firms build cash reserves as a buffer against potential credit supply shocks. They also examine how firms finance attractive investments when they are unable to borrow. More than half of US firms say that rely on internally generated cash flows to fund investment under these circumstances, and about four in ten say they use cash reserves. Not only is investment canceled due to tight credit markets; some firms sell assets to obtain cash. We find that the vast majority of financially constrained firms sold assets in order to fund operations in 2008, while unconstrained firms show no significant propensity  to sell assets. 

They gather firm-level information using a survey of 1050 CFOs (574 in US) conducted in the fourth quarter of 2008. The survey approach provides the opportunity to directly ask managers whether their decisions have been constrained by the cost or availability of credit. They investigate the relation between firm characteristics and whether managerial policies are influenced by access to credit. Using December 2008 as a reference point, they study planned changes for the following 12 months relative to the previous 12 months. Companies planned reductions across all expenditure categories and expected smaller cash reserves and dividend cuts for 2009. 

They ask CFOs to elaborate on the types of frictions they have encountered when trying to raise external finance during the crisis. Among the constrained firms, 80\% experienced less access to credit and 60\% experienced higher cost of funds. Moreover, they use a matching approach (Abadie and Imbens, 2002) where for every firm identified as financially constrained, they find an unconstrained match that is in the same categories, industry and survey quarter. The procedure then estimates the differences in corporate policies for unconstrained firms relative to those that are unconstrained, conditional on being observationally identical. Firms that report themselves as being financially constrained systematically planned to invest less, reduce employment and conserve less cash (-3\%) and pay fewer dividends. These numbers are economically and statistically significant. In particular, their expected cash burn differential (or dissaving) is nearly three times larger during the crisis. According to their survey, the cash holdings of constrained and unconstrained firms in the US were roughly similar one year prior the financial crisis. The 2008 crisis did not affect unconstrained firms' cash levels, but constrained firms burned through a substantial fraction of their cash reserves by year-end 2008. Cash reserves at constrained companies fell by one-fifth, from about 15\% to about 12\% of book assets. This differences between the two groups are highly statistically significant. Since the cash holdings of unconstrained firms stay constant, our test suggests that financially constrained firms have been forced to draw down their cash reserves to cope with the financial crisis. In particular, there is a pronounced reduction in cash levels among financially constrained firms over previous year. This magnitude is startling when combined with our previous result that constrained firms expected to burn through another 15\% of cash holdings during 2009. The Abadie and Imbens (2002) estimator suggests that cash holdings of constrained firms are 2.8\% percentage points lower than that of unconstrained firms following the crisis.  

A question of much debate in the literature concerns the degree wo which firms use internal funds to finance investment when they face credit frictions (see Stein, 2003).They directly ask managers whether they use their firms' internal resources to finance profitable investment opportunities when access to external credit is limited. Their results indicates that firms across all categories are likely to use internal source of funding for their investment when access to external capital markets is limited. Results support the notion that, in the face of a negative credit supply shock, companies consider their internal resources - both operation income and cash savings - as a way to finance future investment. Assuming that firms would prefer to draw on their cash reserves before canceling their planned investments (which is presumably a very costly course of action), we further condition the decision to cancel investment on whether CFOs indicate they are able to use cash to fund investment if external financing sources are inadequate. For those constrained firms for which using internal cash is not an option (perhaps because cash stocks are already depleted) the rate to investment cancellation goes up to 71\% in the US. Similar results imply that constrained firms would opt for selling assets to finance current investment. 

\section{Kaplan and Zingales (1997) - QJE}

They analyze the sample of 49 low-dividend paying firms in Fazzari, Hubbard, and Petersen (1988) from 1970 to 1984. For each firm they collect data from several sources in order to have all the possible information for each firm. Their main finding is that more financially constrained firms display less investment-to-cash flow sensitivity. However, what is important for my own project is that they find the opposite for cash reserves. In contrast, the sensitivity of investment to cash stock now increases with the degree of financing constraints. These latter results, however, are not statistically significant; none of the coefficients are statistically different from each other. The results for investment-cash stock sensitivities are mixed. Over the entire sample period, investment-cash stock sensitivities increase significantly with the degree of financing constraints. However, this pattern does not hold for either the 1970-1977 or the 1978-1984 subperiod. 

\section{Riddick and Whited (2009) - Journal of Finance}

\textbf{Theory.} Savings policies matter for all firms because managers must evaluate the trade-off between using internal and external funds to finance current and future investment. The firm's optimal saving policy depends not only on the cost of external finance, but also on the firm's expected future financing needs, which, in turn, depend on the firm's technology and especially on the uncertainty it faces. The firm's optimal level of cash increases with the cost of external finance. In comparison with a low cost firm, a high cost firm therefore has more slack with which to respond to profit shocks, and it saves or dissaves more aggressively to counteract part of the effects of these shocks (Cash reserves as a buffer to negative profits shocks). If the firm faces an uncertainty environment, it expects to tap external finance more often, and it holds higher cash balances. The intuition about the effect of uncertainty is also evident in the second panel, which depicts a positive relation between cash holdings and the variance of productivity. This increase has also a real option interpretation in which a higher variance leads to a higher option value of cash balances.

\textbf{Comment.} The paper is not super related but you can use it to say that in a theoretical model that keeps into account both financial frictions and uncertainty an uncertainty shock is related to an increase in cash reserves.


\section{Ajello (2016) - AER}

He focuses on US corporations and refer to the sample period from 1989Q1 to 2008Q2. He concentrates on Compustat quarterly data. He defines financial gap (FG) as the extra investment not financed with cash flow. He defines financial gap share (FGS) as FG over total investment. The larger is FGS, the larger firms rely on finance above current cash flow. When the financial gap is negative, internal sources of finance explains up to 22\% of the FGS (liquidation of assets: 3\%; and decrease in cash reserves: 19\%). The liquidation share (the part of internal source of finance used to finance the financial gap) is procyclical which means that firms sell assets and decrease cash during downturns. 

In his model, he assumes that entrepreneurs incur negative financing gaps [...] and that credit and liquidity constraints influence the size and the funding composition of the gaps. 

\textbf{Comment.} Here you can just say that he shows with firm-level data that investment is finance with changes in cash holdings above the current cash flow. This share tends to be larger during crisis and/or credit crunch using data up to the crisis.
















}
\end{document}

