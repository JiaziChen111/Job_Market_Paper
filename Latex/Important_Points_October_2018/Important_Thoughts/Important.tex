\documentclass{article}
\usepackage[utf8]{inputenc}
\usepackage[english]{babel}
\usepackage[a4paper,top=3cm,bottom=3cm,left=3cm,right=3cm,%
bindingoffset=0mm]{geometry}
\usepackage{amssymb}
\usepackage{amsmath}
\newtheorem{prop}{Proposition}
\newtheorem{lemma}{Lemma}
\newenvironment{proof}[1][Proof]{\begin{trivlist}
\item[\hskip \labelsep {\bfseries #1}]}{\end{trivlist}}
\newcommand{\qed}{\nobreak \ifvmode \relax \else
      \ifdim\lastskip<1.5em \hskip-\lastskip
      \hskip1.5em plus0em minus0.5em \fi \nobreak
      \vrule height0.75em width0.75em depth0em\fi}
\usepackage{tikz}
\usepackage{graphicx}
\usepackage{rotating}
\usepackage{float}
\linespread{1.3}
\raggedbottom




%
\font\reali=msbm10 at 12pt
% subsets of real numbers
\newcommand{\real}{\hbox{\reali R}}
\newcommand{\realp}{\hbox{\reali R}_{\scriptscriptstyle +}}
\newcommand{\realpp}{\hbox{\reali R}_{\scriptscriptstyle ++}}
\newcommand{\R}{\mathbb{R}}
\DeclareMathOperator{\E}{\mathbb{E}}
%

\title{Important Thoughts}
\author{Marco Brianti}
\date{A.Y. 2018/2019}

\begin{document}
	\large{

\maketitle

\section*{10/11/2018}

\begin{itemize}
	\item Steady state debt-to-assets ration is about 0.372.
	\item Drop in loans on the 2007 US recession was 26\%.
	\item Thinking about a sign restriction on dispersion? Could it be interesting?
	\item Most of the action in the short run is due to financial shocks. This is consistent with Gilchrist, Sim, and Zackrajsek (2014).
	\item Uncertainty shocks take times to build a large negative impact but they are very persistent. This is consistent with the findings of Jurado, Ludvigson and Ng (2015).
	\item When I observe an uncertainty shock, spillovers on the credit supply side are allowed but the identification assumption is that what can potentially happen in future remains more powerful than what is happening today in response to it. To make it simply, when an unanticipated financial shock hits the economy, firms expect that such an effect will be lower tomorrow. Conversely, when an uncertainty shock hits the economy, firms expect that what might potentially happen tomorrow remains worse than any spillover effect they experience today. In other words, uncertainty shocks should be also related to the risk to be even more financially constrained in future while unanticipated financial shocks should be related to the expectation to be less financially constrained in future. Not surprisingly, an uncertainty shock is highly correlate with an expected financial shock which does not move credit spread today. Although there is much more going on when an uncertainty shock hits the economy we can still think of it as a future financial shock. Indeed, an exercise which should convince the audience of the reliability of my assumption is the correlation between an uncertainty shock and an expected financial shock. Identify an uncertainty shock using a Cholesky identification where uncertainty is ordered first. Identify an expected financial shock as the one orthogonal to current credit spread which maximizes the variance explained of credit spread over the first quarter. The correlation between these shocks is always above 80\%.
	\item Harford, Klasa, Maxwell (2013) is potentially an interesting paper related to the previous point. They find that firms mitigate refinancing risks by increasing their cash holdings by saving from cash flow. It might be useful to discuss propagation effects. Uncertainty affects much more the risk of being financially constrained in future rather than being financially constrained today! Key point I have to make! Uncertainty remember more expected financial shocks.
	\item Why using cash holdings? Because it is an unconditional buffer available in both good and bad states of the world. It is not subject to general equilibrium effects like other financial assets or cash flow. 
	\item Empirical evidence is important because I use results which exploit cross-sectional variation to infer aggregate results in time variation.
	\item In Gilchrist, Sim, and Zakrajsek (2014) firms do not take into account the inter-temporal trade off between investing today versus investing tomorrow because the option of moving resourcing between today and tomorrow is not available. In other words cash flow is not designed in our model. Moreover, they model an uncertainty shock to be observationally equivalent to an unexpected financial shock because without financial frictions an uncertainty shock has basically zero effect in the economy. I am not against the idea that after an uncertainty shock financial frictions are able to amplify its effect but it is not true in the data that controlling for financial frictions the effect is ten times smaller. 
\end{itemize}


}
\end{document}

