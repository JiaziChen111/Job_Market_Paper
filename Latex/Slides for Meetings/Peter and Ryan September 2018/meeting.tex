\documentclass[hyperref={pdfpagelabels=false}]{beamer}
\usepackage{graphicx,lmodern,subfigure,ulem,color,graphicx,tikz,booktabs,natbib}
\usepackage{mathrsfs}
\usetheme{Warsaw}
%\definecolor{beamer@blendedblue}{rgb}{0.1,0.5,0.1}
%\definecolor{ForestGreen}{RGB}{60, 140, 60}
%\setbeamercolor{structure}{fg=beamer@blendedblue}
\setbeamertemplate{navigation symbols}{}
\setbeamertemplate{footline}[frame number]
\bibliographystyle{chicago}
\newcommand{\spitem}{\vspace{.3cm}\item}
\newcommand{\elas}{$E_{labor}$}
\def \ourFigPath {../../} 


\title{Research Proposal}
\author{Marco Brianti}
\institute{Boston College}
\date{September 2018}


\begin{document}
	
	\frame{\titlepage \begin{center} Dissertation Workshop \end{center} }
	
	\frame{\frametitle{Uncertainty as a theoretical concept}		
		
		\begin{itemize}
			
			\item Frank Knight in 1921 defined \textbf{uncertainty} as people's inability to forecast the likelihood of events happening.
			
			\
			
			\
		
		    \item Since Bloom (2009) uncertainty is formally thought as a \textbf{second-moment shock}.
		    
		    \
		    
		    \
		    
		    \item Jurado et al. (2015) formalizes that uncertainty is represented by the volatility of a variable controlling for the \textbf{forecastable component}.
		   
		
		\end{itemize}			
}

	\frame{\frametitle{Uncertainty as an empirical measure}
	
	
\begin{itemize}
	
	\item Uncertainty cannot be directly observed
	
	\
	
	\
	
	\item A series of different proxies
	
	\
	
	\
	
	\item Jurado et al. (2015) provided a generalized measure of macro uncertainty which is consistent with its theoretical concept.
	
	
\end{itemize}	



	
	
	
	
}
	

\end{document}