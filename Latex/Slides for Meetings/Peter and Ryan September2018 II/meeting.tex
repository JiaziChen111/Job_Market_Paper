\documentclass[hyperref={pdfpagelabels=false}]{beamer}
\usepackage{graphicx,lmodern,subfigure,ulem,color,graphicx,tikz,booktabs,natbib}
\usepackage{mathrsfs}
\usetheme{Warsaw}
%\definecolor{beamer@blendedblue}{rgb}{0.1,0.5,0.1}
%\definecolor{ForestGreen}{RGB}{60, 140, 60}
%\setbeamercolor{structure}{fg=beamer@blendedblue}
\setbeamertemplate{navigation symbols}{}
\setbeamertemplate{footline}[frame number]
\bibliographystyle{chicago}
\newcommand{\spitem}{\vspace{.3cm}\item}
\newcommand{\elas}{$E_{labor}$}
%\def \FigPath {Users\th3\Documents\Job_Market_Paper\Code\Figures} 


\title{Research Proposal}
\author{Marco Brianti}
\institute{Boston College}
\date{September 2018}


\begin{document}
	
	\frame{\titlepage \begin{center} Dissertation Workshop \end{center} }
	
	
		\frame{\frametitle{Two Possible Avenues}
		
		
		\begin{enumerate}
			\item News-noise driven uncertainty
			
			\
			
			\item Financial Shocks vs Uncertainty Shocks
		\end{enumerate}
		
		
	}
	
		\frame{\frametitle{Variables of Interest}
		
\begin{eqnarray*}
	\begin{aligned}
\text{Cash Flow} =& \ \text{Undistributed Corporate Profits} \\
          +& \ \text{Consumption of Fixed Capital}    \\
          -& \ \text{Net Capital Transfers Paid}
          \end{aligned}
\end{eqnarray*}

\

\begin{eqnarray*}
	\begin{aligned}
		\text{Undistributed Corporate Profits} =& \ \text{Corporate Profits} - \text{Dividends}    
	\end{aligned}
\end{eqnarray*}

\

where \textbf{consumption of fixed capital} can be simply interpreted as capital depreciation and \textbf{net capital transfers paid} are unrequited transfers, e.g. charity. 


	}
	
	\frame{\frametitle{Key Variable}	
		
		
		\begin{eqnarray*}
			\begin{aligned}
				\frac{\text{Cash Flow}}{\text{Corporate Profits}} =& \ 1 - \frac{\text{Dividends + Consumption of K - K Transfers}}{\text{Corporate Profits}} \\
				=&  \ 1 - \frac{\text{Cash Dispersion}}{\text{Corporate Profits}}
			\end{aligned}
		\end{eqnarray*}	
	
	\
	
	
	\textbf{Cash Flow} is a profit-related measure of internal funds available for investment. [The NIPA Handbook, December 2015]
	
	
	

		
	}
	\frame{\frametitle{Economic Intuition}
	
	
\begin{itemize}
	
	\item 	It is reasonable to think that after a \textbf{decrease in credit supply}, firms would attempt to decrease cash dispersion relatively to corporate profits $\Rightarrow$ increase cash flow relatively to current profits. 
	
	\
	
	\
	
	\item It is reasonable to think that after an \textbf{increase in uncertainty}, corporate profits would decrease but firms would not attempt to decrease cash dispersion relatively to corporate profits since they do not need to increase cash flow relatively to current profits.
	
	\begin{itemize}
		\item As a result an \textbf{uncertainty shock} should decrease cash flow relatively to current profits.
	\end{itemize} 
	

	
	
\end{itemize}

}	

\frame{\frametitle{Suggestive Evidence}
	
	Run the following regression,
	
	\
	
	$$
	\frac{CF_t}{CP_t} = \alpha + B(L)X_{t-1} + \beta^{F} F_t + \beta^{U} U_t + \varepsilon_t
	$$
	
	\
	
	where $CF_t$ and $CP_t$ are cash flow and corporate profits as described above, $X_{t-1}$ is a vector of control variables,
	
	\
	
	$$
	X_{t-1} = [GDP_{t-1} \ I_{t-1} \ C_{t-1} \ SP_{t-1} \ H_{t-1} \ U_{t-1} \ F_{t-1} \ CF_{t-1}/CP_{t-1}]
	$$
	
	\
	
	and $F_t$ and $U_t$ are proxies for financial shocks and uncertainty shocks, respectively. 

	
}

	
	\frame{\frametitle{Results}
		

Benchmark regression,
$$
\frac{CF_t}{CP_t} = \alpha + B(L)X_{t-1} + \beta^{F} F_t + \beta^{U} U_t + \varepsilon_t
$$

\

\

\begin{itemize}
	\item $\beta^F$ is always positive and significant at $1$\%.
	
	\
	
	\
	
	\item $\beta^U$ is either negative and significant at $10$\% or not significant. 
\end{itemize}

			





		
	}



	\frame{\frametitle{Interpretation - Erosion of the Financial System}
		
Graph
	
		
	}

	\frame{\frametitle{Interpretation - Erosion of the Financial System}
		
Graph
		
	}

	\frame{\frametitle{Technically Speaking (I)}
		
		Assume you use OLS techniques to regress $X_t$ on its own past
		$$
		X_t = B_1 X_{t-1} + B_2 X_{t-2} + \dots + B_p X_{t-p} + \iota_t
		$$
		where $X_t = [U_t \ \  Y_t  \ \  F_t]'$, $U_t$ represents a proxy for uncertainty, $Y_t$ a column vector of macro variables, and $F_t$ a vector of financial variables.
		
		\
		
		Moreover, $\iota_t = [\iota^U_t \ \ \iota^Y_t \ \ \iota^F_t]'$ is a vector of time-varying innovations related to the corresponding variables.
		
		\
		
		In general, $\iota_t$ does not represent a vector of structural shocks since 
		$$
		\iota_t \iota_t' \neq I_n
		$$
		which implies that innovations represent a (linear) combination of the structural shocks.
	}

	\frame{\frametitle{Technically Speaking (II)}
	
	
	Structural VARs methods aim to solve the following system in order to recover structural shocks
	$$
	\iota_t = C s_t \ \ \Rightarrow \ \ s_t = C^{-1} \iota_t \ \ \Rightarrow \ \ s_t = A \iota_t
	$$
	which is
	$$
    \begin{cases}
	s_t^U = A_{11} \iota_t^U + A_{12} \iota_t^Y + A_{13} \iota_t^F \\
	s_t^Y = A_{21} \iota_t^U + A_{22} \iota_t^Y + A_{23} \iota_t^F \\
	s_t^U = A_{31} \iota_t^U + A_{32} \iota_t^Y + A_{33} \iota_t^F
	\end{cases}
$$

\begin{enumerate}
	\item \textbf{Latent variable} $\Rightarrow$ $\iota_t^U$ may not represent innovations to uncertainty
	
	
	\item \textbf{Simultaneity} $\Rightarrow$ Each element of $A$ is different from zero
	
	\item \textbf{Reverse causality} $\Rightarrow$ $\iota_t^U$ may be lead by $s_{t,t+h}$, $h > 0$
	
	
	\item \textbf{Financial shocks} $\Rightarrow$ $E [ \iota_t^U  \iota_t^{F'}]  \neq 0$ and large
	
\end{enumerate}

}


	\frame{\frametitle{(1) Latent Variable}
		
		Not surprisingly, $Corr(VXO_t, JLN_t) = 0.4139$
		
		\
		
		However, $Corr(\iota_t^{VXO},\iota_t^{JLN}) \in [-0.1865 \ \ 0]$
	
\

Which means that although the 2 raw series are highly correlated, once we control for available information at $t-1$ then they convey different information.

\

\textbf{Solution.} JLN proxy is consistent with the theoretical definition of uncertainty. 

\

$\Rightarrow$ VXO measures \textbf{macro volatility} and not macro uncertainty.

	
}




	\frame{\frametitle{(2) Simultaneity with other shocks}

	
In general,	
$$
corr(\iota^{JLN}_t,s^Y_t) \approx 0
$$
which implies that uncertainty innovations are fairly uncorrelated with macro structural shocks series derived in the literature.

\

$s_t^Y$ are several series of macro structural shocks derived by the literature (possibly via narrative approach). 
\begin{itemize}
	\item Romer and Romer (2010) unanticipated tax shocks
	\item Martens and Ravn (2011) labor productivity shocks
	\item Leeper et al. (2013) anticipated tax shocks
	\item Kilian (2009) oil shocks
	\item \dots
\end{itemize}
}


	\frame{\frametitle{(3) Reverse causality with news shocks}
		
		\begin{itemize}
			
			\item \textbf{JLN proxy} controls for the forecastable part of each variable
			
			\
			
			
			\item Some structural shocks shown above are \textbf{anticipated}
			
			\
			
			
			\item We can possibly control for \textbf{news shocks} to TFP
			\begin{itemize}
				\item However, we will have to assume that TFP is fully exogenous
			\end{itemize}
			
			\
			
			
			\item \textbf{Surveys} can help for the short run horizon
			\begin{itemize}
				\item SPF has the best timing
			\end{itemize}
		
		\
		
		
			\item Most importantly, we should control for the shocks and the \textbf{square of the shocks}
			\begin{itemize}
				\item Potentially, uncertainty may evenly react for large shocks no matter the sign
			\end{itemize}
		
		
		\end{itemize}
	
}

	\frame{\frametitle{(4) Financial Shocks vs Uncertainty Shocks}
	
Stock and Watson (2012); Caldara, Fuentes-Albero, Gilchrist, and Zakrajzek (2016) shown that uncertainty shocks and financial shocks are deeply confounded.
$$
corr(\iota^{EBP}_t,\iota^{JLN}_t) \approx 0.45
$$
where $\iota^{EBP}_t$ is an innovation in the \textbf{excess bond premium} from Gilchrist and Zakrajzek (2012).

\

\

Literature did not succeed yet to disentangle the two exogenous sources:
\begin{itemize}
	\item \textbf{External instruments} do not seem to be available
	\item \textbf{Internal instruments} are difficult to find because variables respond analogously to both shocks
\end{itemize}
	
}



	\frame{\frametitle{(4) Financial Shocks vs Uncertainty Shocks - Solution (I)}
		
I propose a \textbf{novel family of internal instruments} which can help out to disentangle the two exogenous shocks.	

\

\textbf{Economic Intuition.} 

\begin{itemize}
	\item An exogenous deterioration of credit conditions should display the attempt of borrowers to fund their projects with \textbf{alternative sources} (at least on impact): internal cash flow, equity issuance, ...
	\item Alternatively, following real-options models (Bernanke, 1983; Brennan and Schwartz, 1985; McDonald and Siegel, 1986) after an uncertainty shock firms prefer to \textbf{wait-and-see} without undertake any investment.
\end{itemize}

	
	}


	\frame{\frametitle{(4) Financial Shocks vs Uncertainty Shocks - Solution (II)}
	
	Although the impact effect on investment is expected to be negative in both cases, I expect
	\begin{itemize}
		\item a financial shock to have a \textbf{negative impact} on internal cash flow;
		\item an uncertainty shock to have a \textbf{non-negative impact} on internal cash flow.
	\end{itemize}

\

\
	
The two shocks can be disentangled via \textbf{sign restrictions} \`a la Uhlig (2005)
	
	
}


\end{document}