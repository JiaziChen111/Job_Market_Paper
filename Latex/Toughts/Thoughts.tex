\documentclass{article}
\usepackage[utf8]{inputenc}
\usepackage[english]{babel}
\usepackage[a4paper,top=3cm,bottom=3cm,left=3cm,right=3cm,%
bindingoffset=0mm]{geometry}
\usepackage{amssymb}
\usepackage{amsmath}
\newtheorem{prop}{Proposition}
\newtheorem{lemma}{Lemma}
\newenvironment{proof}[1][Proof]{\begin{trivlist}
\item[\hskip \labelsep {\bfseries #1}]}{\end{trivlist}}
\newcommand{\qed}{\nobreak \ifvmode \relax \else
      \ifdim\lastskip<1.5em \hskip-\lastskip
      \hskip1.5em plus0em minus0.5em \fi \nobreak
      \vrule height0.75em width0.75em depth0em\fi}
\usepackage{tikz}
\usepackage{graphicx}
\usepackage{rotating}
\usepackage{float}
\linespread{1.3}
\raggedbottom




%
\font\reali=msbm10 at 12pt
% subsets of real numbers
\newcommand{\real}{\hbox{\reali R}}
\newcommand{\realp}{\hbox{\reali R}_{\scriptscriptstyle +}}
\newcommand{\realpp}{\hbox{\reali R}_{\scriptscriptstyle ++}}
\newcommand{\R}{\mathbb{R}}
\DeclareMathOperator{\E}{\mathbb{E}}
%

\title{Job Market Paper}
\author{Marco Brianti}
\date{A.Y. 2018/2019}

\begin{document}
	\large{

\maketitle

\section*{7/27/2018 - Noise Shocks and Heterogeneous Agents}

Initial idea is to identify a new type of shocks called \textbf{idiosyncratic noise shocks} which needs to be compared to the old class of \textbf{aggregate noise shocks}. 

An \textbf{aggregate noise shock} as defined by Lorenzoni (2009) is a \textit{noisy public signal regarding aggregate productivity}. This first type of shocks assume that agents coordinate their choices due to an aggregate biased signal. In this case, the underlying assumption is on the quality of the signal rather than the quantity. Information is strong and clear since coordinates agents' choices but it does not correctly predict future economic outcomes.

An \textbf{idiosyncratic noise shock} instead is defined as a \textit{set of noisy signals which spread out expectations regarding aggregate productivity}. This second type of shocks assume that agents do not coordinate their choices because this noise does not turn into an unique signal which equally hits the information set of each agent. In this case, the underlying assumption is on the quantity of the signal rather than the quality. At this stage, I want to remain agnostic on the quality of the signal, but I assume that information is quantitatively weak since agents are not anymore able to coordinate their choices due to a clear unique signal. 

Both of them take part in the larger family of shocks (hereafter \textbf{forecast error shocks}) which in common share the fact that in both cases agents commit mistakes. In the case of \textbf{aggregate noise shocks} agents take the same wrong choices while in the case of \textbf{idiosyncratic noise shocks} agents take different incorrect decisions. In order to clarify this last concept I need to formalize a little bit more how we should mathematically define both shocks.

Consider a simple economy populated by $I$ agents where each agent $i$ attempts to correctly forecast fundamental variable $x_{t+1}$ given the information set at time $t$. I define a forecast error shock $\phi_t$ as follows,
\begin{eqnarray}\label{eq:FE_shock}
\phi_t = I^{-1} \sum_{i=1}^I \big\{ E_t^i [ x_{t+1} ] - x_{t+1} \big\}^2 
\end{eqnarray}
where $\E^i_t [ x_{t+1} ]$ is the expectation of agent $i$ of fundamental outcome $x_{t+1}$ given the information set at time $t$ and $\E^i_t [ x_{t+1} ] - x_{t+1}$ is the forecast error of agent $i$ of fundamental outcome $x_{t+1}$ given the same information. Thus, $\phi_t$ represents the mean squared forecast error across agents in the whole economy. In other words, $\phi_t$ is an index which broadly represents the precision of expectations of agents in the economy at time $t$. Now, the interesting part of \ref{eq:FE_shock} is that can be decomposed as follows,
\begin{eqnarray}\label{eq:decomposition}
\begin{aligned}
\phi_t &= I^{-1} \sum_{i=1}^I \big\{ E_t^i ( x_{t+1} ) - x_{t+1} \big\}^2 \\
&= I^{-1} \sum_{i=1}^I \big\{  E_t^i ( x_{t+1} )^2 - 2E_t^i ( x_{t+1} )x_{t+1} + x_{t+1}^2  \big\} \\
&= I^{-1} \sum_{i=1}^I E_t^i ( x_{t+1} )^2 - 2\bar{x}_{t,t+1}^i x_{t+1} + x_{t+1}^2 \\
&= I^{-1} \sum_{i=1}^I E_t^i ( x_{t+1} )^2 - (\bar{x}_{t,t+1}^i)^2 + (\bar{x}_{t,t+1}^i)^2 - 2\bar{x}_{t,t+1}^i x_{t+1} + x_{t+1}^2 \\
&= Var^i_t(x_{t+1}) + (\bar{x}_{t,t+1}^i - x_{t+1})^2, \\
\end{aligned}
\end{eqnarray}
where $\bar{x}_{t,t+1}^i = I^{-1} \sum_{i=1}^I E_t^i ( x_{t+1} )$ is the average expectation across agents of $x_{t+1}$ given the information set at time $t$. Moreover, notice that Equation \ref{eq:decomposition} is divided into two parts: (i) $Var^i_t(x_{t+1})$ which is the variance across agents of expectations of $x_{t+1}$ given information set at time $t$; and (ii) $(\bar{x}_{t,t+1}^i - x_{t+1})^2$ which is the square of difference between the average expectation across agents of $x_{t+1}$ at time $t$ and its actual realization.

Not surprisingly, we define an \textbf{idiosyncratic noise shock} $\iota_t$ as follows
\begin{eqnarray}\label{eq:idio_shock}
\iota_t = Var^i_t(x_{t+1})
\end{eqnarray}
and an \textbf{aggregate noise shock} as follows
\begin{eqnarray}\label{eq:aggre_shock}
\eta_t = \bar{x}_{t,t+1}^i - x_{t+1}
\end{eqnarray}
\textbf{Idiosyncratic noise shocks} represent situations where agents are not able to coordinate their choices because they expect different future outcomes. In this case, the source of noise is due to a lack of an unique and clear signal rather than to a bias on the signal per se. On the other hand, \textbf{aggregate noise shocks} represent situations where agents coordinate over a choice which is not aligned with the future realized outcome. In this case, the signal is clear to everyone and the source of noise is due to a bias in the information received.

Then, $\phi_t$ can be represented as
\begin{eqnarray}\label{eq:bias_var_dec}
\phi_t = \iota_t + \eta_t^2
\end{eqnarray}
which is simply the Variance-Bias decomposition of a mean square forecast error shock $\phi_t$. Goal of the project would be to identify both $\iota_t$ and $\eta_t$ to quantify the effects of this two different types of shocks.










}
\end{document}

